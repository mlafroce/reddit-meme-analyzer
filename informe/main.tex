\documentclass[12pt, a4paper]{article}

\usepackage[utf8]{inputenc} % Especifica la codificación de caracteres de los documentos.
\usepackage[spanish]{babel} % Indica que el documento se escribirá en español.
\usepackage{graphicx,grffile}
\graphicspath{{build/img/}}

\usepackage{anyfontsize}
\usepackage{array}
\usepackage{amssymb,amsmath} % utilizado por pandoc para renderizar formulas
\usepackage{caption}
\usepackage{float}
\usepackage{hyperref}
\usepackage{longtable}
\usepackage{multirow}
\usepackage{mdframed}
\usepackage{titlesec}
\usepackage{todonotes}

\usepackage{./custom}

\begin{document}

\begin{titlepage}
	\centering
	\vspace{1cm}
	{\bfseries\LARGE Universidad de Buenos Aires \par}
	{\scshape\Large Facultad de Ingeniería\par}
	\vspace{3cm}
	{\scshape\Large 75.74 Sistemas distribuidos\par}
	\vspace{3cm}
	{\scshape\Huge Trabajo práctico N° 2 \par}
	{\scshape\Huge Reddit meme analyzer \par}
	\vfill
	{\Large Autor: \par}
	{\Large Matías Lafroce \par}
	\vfill
	{\Large 21 de junio 2022 \par}
\end{titlepage}

\tableofcontents

\include{build/informe}

\end{document}
